% This LaTeX file is self-contained and includes all necessary modifications.
% It uses the 'filecontents*' environment to create the .bib file automatically.
% Compile with: pdfLaTeX -> BibTeX -> pdfLaTeX -> pdfLaTeX

\begin{filecontents*}[overwrite]{Quantum-Reality.bib}
@book{Kuhn1962,
  author    = {Kuhn, Thomas S.},
  title     = {{The Structure of Scientific Revolutions}},
  publisher = {University of Chicago Press},
  year      = {1962}
}

@article{Beauregard2014,
  author  = {Beauregard, Mario and Schwartz, Gary E. and Miller, Lisa and Dossey, Larry and Moreira-Almeida, Alexander and Schlitz, Marilyn and Sheldrake, Rupert and Tart, Charles},
  title   = {{Manifesto for a Post-Materialist Science}},
  journal = {EXPLORE: The Journal of Science and Healing},
  volume  = {10},
  number  = {5},
  pages   = {272--274},
  year    = {2014},
  doi     = {10.1016/j.explore.2014.06.008}
}

@book{Nagel2012,
  author  = {Nagel, Thomas},
  title   = {{Mind and Cosmos: Why the Materialist Neo-Darwinian Conception of Nature Is Almost Certainly False}},
  publisher = {Oxford University Press},
  year      = {2012}
}

@article{Chalmers1995,
  author  = {Chalmers, David J.},
  title   = {{Facing up to the problem of consciousness}},
  journal = {Journal of Consciousness Studies},
  volume  = {2},
  number  = {3},
  pages   = {200--219},
  year    = {1995}
}

@book{Kirkpatrick2021,
  author  = {Kirkpatrick, Sean},
  title   = {{In Plain Sight: An investigation into UFOs and impossible science}},
  publisher = {John Murray},
  year      = {2021}
}

@incollection{Wheeler1990,
    author    = {Wheeler, John Archibald},
    title     = {{Information, physics, quantum: The search for links}},
    booktitle = {Complexity, Entropy, and the Physics of Information},
    year      = {1990},
    editor    = {Zurek, Wojciech H.},
    publisher = {Addison-Wesley},
    pages     = {3--28}
}

@book{Kastrup2019,
    author    = {Kastrup, Bernardo},
    title     = {{The Idea of the World: A multi-disciplinary argument for the mental nature of reality}},
    publisher = {Iff Books},
    year      = {2019}
}

@article{Alcubierre1994,
    author  = {Alcubierre, Miguel},
    title   = {{The warp drive: hyper-fast travel within general relativity}},
    journal = {Classical and Quantum Gravity},
    volume  = {11},
    number  = {5},
    pages   = {L73--L77},
    year    = {1994},
    doi     = {10.1088/0264-9381/11/5/001}
}

@article{Cronin2009,
    author  = {Cronin, Alexander D. and Schmiedmayer, J{\"o}rg and Pritchard, David E.},
    title   = {{Optics and interferometry with atoms and molecules}},
    journal = {Reviews of Modern Physics},
    volume  = {81},
    pages   = {1051--1129},
    year    = {2009},
    doi     = {10.1103/RevModPhys.81.1051}
}

@techreport{ODNI2021,
    author      = {{Office of the Director of National Intelligence}},
    title       = {{Preliminary Assessment: Unidentified Aerial Phenomena}},
    year        = {2021},
    month       = {June},
    institution = {Office of the Director of National Intelligence},
    type        = {Government Report},
    note        = {Available at: \url{https://www.dni.gov/index.php/newsroom/reports-publications/reports-publications-2021/item/2223-preliminary-assessment-unidentified-aerial-phenomena}}
}

@article{Schmidt1986,
    author  = {Schmidt, R. O.},
    title   = {{Multiple emitter location and signal parameter estimation}},
    journal = {IEEE Transactions on Antennas and Propagation},
    volume  = {34},
    number  = {3},
    pages   = {276--280},
    year    = {1986},
    doi     = {10.1109/TAP.1986.1143830}
}

@article{Kalman1960,
    author  = {Kalman, R. E.},
    title   = {{A New Approach to Linear Filtering and Prediction Problems}},
    journal = {Journal of Basic Engineering},
    volume  = {82},
    number  = {1},
    pages   = {35--45},
    year    = {1960},
    doi     = {10.1115/1.3662552}
}

@article{Bell1964,
    author  = {Bell, J. S.},
    title   = {{On the Einstein Podolsky Rosen Paradox}},
    journal = {Physics Physique Fizika},
    volume  = {1},
    pages   = {195--200},
    year    = {1964}
}

@book{Popper1959,
    author    = {Popper, Karl R.},
    title     = {{The Logic of Scientific Discovery}},
    publisher = {Hutchinson},
    year      = {1959}
}

@book{Radhakrishnan1953,
    author    = {Radhakrishnan, Sarvepalli},
    title     = {{The Principal Upanishads}},
    publisher = {Harper \& Brothers},
    year      = {1953}
}

@article{Rovelli1996,
    author  = {Rovelli, Carlo},
    title   = {{Relational Quantum Mechanics}},
    journal = {International Journal of Theoretical Physics},
    volume  = {35},
    pages   = {1637--1678},
    year    = {1996},
    doi     = {10.1007/BF02302261}
}

@book{Bar-Shalom2001,
    author    = {Bar-Shalom, Yaakov and Li, X. Rong and Kirubarajan, T.},
    title     = {{Estimation with Applications to Tracking and Navigation: Theory, Algorithms, and Software}},
    publisher = {Wiley-Interscience},
    year      = {2001}
}

@article{Kass1995,
    author  = {Kass, Robert E. and Raftery, Adrian E.},
    title   = {{Bayes Factors}},
    journal = {Journal of the American Statistical Association},
    volume  = {90},
    number  = {430},
    pages   = {773--795},
    year    = {1995},
    doi     = {10.1080/01621459.1995.10476572}
}

@book{Dasgupta1922,
    author    = {Dasgupta, Surendranath},
    title     = {{A History of Indian Philosophy, Vol. 1}},
    publisher = {Cambridge University Press},
    year      = {1922}
}

@article{EPR1935,
    author  = {Einstein, A. and Podolsky, B. and Rosen, N.},
    title   = {{Can Quantum-Mechanical Description of Physical Reality Be Considered Complete?}},
    journal = {Physical Review},
    volume  = {47},
    number  = {10},
    pages   = {777--780},
    year    = {1935},
    doi     = {10.1103/PhysRev.47.777}
}

@article{Bucar2022,
    author  = {Bucar, Tim and Bongs, Kai and Holynski, Michael and Vovrosh, Juraj and Metje, Nicole and Chamberlain, Andrew},
    title   = {{Advances in Portable Atom Interferometry-Based Gravity Sensing}},
    journal = {Sensors},
    volume  = {22},
    number  = {19},
    pages   = {7156},
    year    = {2022},
    doi     = {10.3390/s22197156}
}

@inproceedings{Carion2020,
    author    = {Carion, Nicolas and Massa, Francisco and Synnaeve, Gabriel and Usunier, Nicolas and Kirillov, Alexander and Zagoruyko, Sergey},
    title     = {{End-to-End Object Detection with Transformers}},
    booktitle = {Computer Vision -- ECCV 2020},
    year      = {2020},
    publisher = {Springer International Publishing},
    pages     = {213--229},
    doi       = {10.1007/978-3-030-58452-8_13}
}

@article{Patty2022,
    author  = {Patty, C. H. Lucas and Pommerol, Antoine and K{\"u}hn, Jonas G. and Demory, Brice-Olivier and Thomas, Nicolas},
    title   = {{Directional Aspects of Vegetation Linear and Circular Polarization Biosignatures}},
    journal = {Astrobiology},
    volume  = {22},
    number  = {9},
    pages   = {1034--1046},
    year    = {2022},
    doi     = {10.1089/ast.2021.0156}
}

@article{Han2022,
    author  = {Han, Biao and Li, Run-Bing and Wang, Jin and Wu, Jie-Xin and Huang, Pu and Xu, Zhen-Tai and Wang, Jing-Min and Wu, Bin and Zhuang, Wei and Chen, Shuai and Xue, Zhong-Kui and Zhan, Ming-Sheng},
    title   = {{Compact High-Resolution Absolute-Gravity Gradiometer Based on Atom Interferometers}},
    journal = {Physical Review Applied},
    volume  = {18},
    number  = {5},
    pages   = {054006},
    year    = {2022},
    doi     = {10.1103/PhysRevApplied.18.054006}
}

@techreport{AARO2024,
    author      = {{All-domain Anomaly Resolution Office}},
    title       = {{Report on the Historical Record of U.S. Government Involvement with Unidentified Anomalous Phenomena (UAP)}},
    year        = {2024},
    month       = {March},
    institution = {All-domain Anomaly Resolution Office},
    type        = {Government Report},
    note        = {Volume I, Version 2; available at: \url{https://www.aaro.mil/Portals/136/PDFs/UAP_Historical_Record_Report_V2_03_20_24.pdf}}
}

@article{Blevins2019,
    author  = {Blevins, John A. and Young, Larry A. and Re-entrant, H. D. and Griffin, K. L.},
    title   = {{Aerothermodynamics of a HIFiRE-1-Like Vehicle}},
    journal = {AIAA Scitech 2019 Forum},
    year    = {2019},
    doi     = {10.2514/6.2019-2169}
}

@article{Guyon2018,
    author  = {Guyon, Olivier and Martinache, Frantz and Belikov, Ruslan and Mazin, Benjamin and Cady, Eric and Mignant, David and Tuthill, Peter},
    title   = {{High-contrast imaging for space and ground: a review of past, present, and future technologies}},
    journal = {Proceedings of the SPIE},
    volume  = {10703},
    pages   = {107030K},
    year    = {2018},
    doi     = {10.1117/12.2312686}
}

@article{Robbins2013,
    author  = {Robbins, M. S. and Hadwen, B. J.},
    title   = {{The noise performance of scientific CMOS cameras}},
    journal = {IEEE Transactions on Electron Devices},
    volume  = {60},
    number  = {1},
    pages   = {89--96},
    year    = {2013},
    doi     = {10.1109/TED.2012.2227593}
}

@article{Gardner2006,
    author  = {Gardner, William A. and Napolitano, Antonio and Paura, Luigi},
    title   = {{Cyclostationarity: Half a century of research}},
    journal = {Signal Processing},
    volume  = {86},
    number  = {4},
    pages   = {639--697},
    year    = {2006},
    doi     = {10.1016/j.sigpro.2005.06.016}
}

@article{Sikivie2021,
    author  = {Sikivie, P.},
    title   = {{Invisible Axion Search Methods}},
    journal = {Reviews of Modern Physics},
    volume  = {93},
    number  = {1},
    pages   = {015004},
    year    = {2021},
    doi     = {10.1103/RevModPhys.93.015004}
}

@article{Ford1998,
    author  = {Ford, L. H. and Roman, Thomas A.},
    title   = {{The Quantum Interest Conjecture}},
    journal = {Physical Review D},
    volume  = {58},
    number  = {10},
    pages   = {104018},
    year    = {1998},
    doi     = {10.1103/PhysRevD.58.104018}
}

@article{Frauchiger2018,
    author  = {Frauchiger, Daniela and Renner, Renato},
    title   = {{Quantum theory cannot consistently describe the use of itself}},
    journal = {Nature Communications},
    volume  = {9},
    number  = {1},
    pages   = {3711},
    year    = {2018},
    doi     = {10.1038/s41467-018-05739-8}
}

@article{Tononi2016,
    author  = {Tononi, Giulio and Boly, Melanie and Massimini, Marcello and Koch, Christof},
    title   = {{Integrated information theory: from consciousness to its physical substrate}},
    journal = {Nature Reviews Neuroscience},
    volume  = {17},
    number  = {7},
    pages   = {450--461},
    year    = {2016},
    doi     = {10.1038/nrn.2016.44}
}

@techreport{NASA2023,
    author      = {{NASA UAP Independent Study Team}},
    title       = {{Unidentified Anomalous Phenomena: Independent Study Team Report}},
    year        = {2023},
    month       = {September},
    institution = {National Aeronautics and Space Administration},
    type        = {Government Report},
    note        = {Available at: \url{https://science.nasa.gov/wp-content/uploads/2023/09/uap-independent-study-team-final-report.pdf}}
}

@article{Mancini2002,
    author  = {Mancini, S. and Giovannetti, V. and Vitali, D. and Tombesi, P.},
    title   = {{Entangling Macroscopic Motion with Quantum Light}},
    journal = {Physical Review Letters},
    volume  = {88},
    number  = {12},
    pages   = {120401},
    year    = {2002},
    doi     = {10.1103/PhysRevLett.88.120401}
}

@article{Hameroff2014,
    author  = {Hameroff, Stuart and Penrose, Roger},
    title   = {{Consciousness in the universe: A review of the \`Orch OR' theory}},
    journal = {Physics of Life Reviews},
    volume  = {11},
    number  = {1},
    pages   = {39--78},
    year    = {2014},
    doi     = {10.1016/j.plrev.2013.08.002}
}

@article{Hossenfelder2018,
    author  = {Hossenfelder, Sabine},
    title   = {{Screams for Explanation: Finetuning and Naturalness in the Foundations of Physics}},
    journal = {Foundations of Physics},
    volume  = {48},
    pages   = {1443--1463},
    year    = {2018},
    doi     = {10.1007/s10701-018-0209-4}
}

@article{Pfenning1997,
    author  = {Pfenning, Michael J. and Ford, L. H.},
    title   = {{The unphysical nature of "Warp Drive"}},
    journal = {Classical and Quantum Gravity},
    volume  = {14},
    number  = {7},
    pages   = {1743--1751},
    year    = {1997},
    doi     = {10.1088/0264-9381/14/7/011}
}

@article{Roy2017,
    author  = {Roy, S. and Kumar, S. and Krishna, B. and Katiyar, V.},
    title   = {{An application of interacting multiple model algorithm for tracking a highly maneuvering target}},
    journal = {IFAC-PapersOnLine},
    volume  = {50},
    number  = {1},
    pages   = {9992--9997},
    year    = {2017},
    doi     = {10.1016/j.ifacol.2017.08.1506}
}

@article{Hensen2015,
    author  = {Hensen, B. and Bernien, H. and Dr{\'e}au, A. E. and Reiserer, A. and Kalb, N. and Blok, M. S. and Ruitenberg, J. and Vermeulen, R. F. L. and Schouten, R. N. and Abell{\'a}n, C. and Amaya, W. and Pruneri, V. and Mitchell, M. W. and Markham, M. and Twitchen, D. J. and Elkouss, D. and Wehner, S. and Taminiau, T. H. and Hanson, R.},
    title   = {{Loophole-free Bell inequality violation using electron spins separated by 1.3 kilometres}},
    journal = {Nature},
    volume  = {526},
    number  = {7575},
    pages   = {682--686},
    year    = {2015},
    doi     = {10.1038/nature15759}
}

@book{Kaku2021,
    author    = {Kaku, Michio},
    title     = {{The God Equation: The Quest for a Theory of Everything}},
    publisher = {Doubleday},
    year      = {2021}
}

@article{Schlosshauer2019,
    author  = {Schlosshauer, Maximilian},
    title   = {{The quantum-to-classical transition and decoherence}},
    journal = {Reviews of Modern Physics},
    volume  = {91},
    number  = {1},
    pages   = {015001},
    year    = {2019},
    doi     = {10.1103/RevModPhys.91.015001}
}

@book{Stapp2011,
    author    = {Stapp, Henry P.},
    title     = {{Mindful Universe: Quantum Mechanics and the Participating Observer}},
    publisher = {Springer},
    year      = {2011}
}

@book{Jeffreys1961,
  author    = {Jeffreys, Harold},
  title     = {{Theory of Probability}},
  publisher = {Oxford University Press},
  year      = {1961},
  edition   = {3rd}
}

@article{Woollacott2022,
    author  = {Woollacott, Marjorie H. and Shamas, G. and Tressoldi, P. E. and Tarr, J. and Facco, E. and Pederzoli, L. and Moreira-Almeida, A. and Paul, D. J. and Kok, J.},
    title   = {{Neurophysiological correlates of spiritual experiences: A systematic review}},
    journal = {EXPLORE},
    volume  = {18},
    number  = {1},
    pages   = {68--81},
    year    = {2022},
    doi     = {10.1016/j.explore.2020.12.008}
}

@book{Welch1995,
  author    = {Welch, G. and Bishop, G.},
  title     = {{An Introduction to the Kalman Filter}},
  year      = {1995},
  publisher = {University of North Carolina at Chapel Hill, Department of Computer Science},
  type      = {Technical Report},
  number    = {TR 95-041}
}

@article{Engel2007,
    author  = {Engel, Gregory S. and Calhoun, Tessa R. and Read, Elizabeth L. and Ahn, Tae-Kyu and Man{\v{c}}al, Tom{\'a}{\v{s}} and Cheng, Yuan-Chung and Blankenship, Robert E. and Fleming, Graham R.},
    title   = {{Evidence for wavelike energy transfer through quantum coherence in photosynthetic systems}},
    journal = {Nature},
    volume  = {446},
    number  = {7137},
    pages   = {782--786},
    year    = {2007},
    doi     = {10.1038/nature05678}
}

@book{Jaynes2003,
    author    = {Jaynes, E. T.},
    title     = {{Probability Theory: The Logic of Science}},
    publisher = {Cambridge University Press},
    year      = {2003}
}

@article{Gisin2021,
    author  = {Gisin, Nicolas},
    title   = {{Indeterminism in Physics, Classical and Quantum}},
    journal = {Erkenntnis},
    volume  = {86},
    number  = {5},
    pages   = {1249--1262},
    year    = {2021},
    doi     = {10.1007/s10670-019-00144-x}
}

@article{Goff2017,
    author = {Goff, Philip},
    title = {{Panpsychism}},
    journal = {Philosophy Compass},
    volume = {12},
    number = {11},
    pages = {e12450},
    year = {2017},
    doi = {10.1111/phc3.12450}
}

@article{Aharonov1988,
    author = {Aharonov, Y. and Albert, D. Z. and Vaidman, L.},
    title = {{How the result of a measurement of a component of the spin of a spin-1/2 particle can turn out to be 100}},
    journal = {Physical Review Letters},
    volume = {60},
    number = {14},
    pages = {1351--1354},
    year = {1988},
    doi = {10.1103/PhysRevLett.60.1351}
}

@article{Cox1946,
    author  = {Cox, R. T.},
    title   = {{Probability, Frequency and Reasonable Expectation}},
    journal = {American Journal of Physics},
    volume  = {14},
    number  = {1},
    pages   = {1--13},
    year    = {1946},
    doi     = {10.1119/1.1990764}
}

@book{Sivia2006,
    author    = {Sivia, D. S. and Skilling, J.},
    title     = {{Data Analysis: A Bayesian Tutorial}},
    publisher = {Oxford University Press},
    year      = {2006},
    edition = {2nd}
}

@article{Bohr1928,
    author = {Bohr, N.},
    title = {{The Quantum Postulate and the Recent Development of Atomic Theory}},
    journal = {Nature},
    volume = {121},
    number = {3050},
    pages = {580--590},
    year = {1928},
    doi = {10.1038/121580a0}
}

@article{Tegmark2000,
  author    = {Tegmark, Max},
  title     = {{Importance of quantum decoherence in brain processes}},
  journal   = {Physical Review E},
  volume    = {61},
  number    = {4},
  pages     = {4194--4206},
  year      = {2000},
  doi       = {10.1103/PhysRevE.61.4194}
}

@article{Andre2024,
    author    = {Andr{\'e}, Jo{\~a}o and Chaves, Rui and Moreira, Cristiano},
    title     = {{The Quantum Biology of the Brain: An Introduction}},
    journal   = {Entropy},
    volume    = {26},
    number    = {5},
    pages     = {384},
    year      = {2024},
    doi       = {10.3390/e26050384}
}

@incollection{Rovelli2021,
    author    = {Rovelli, Carlo},
    title     = {{Relational Quantum Mechanics}},
    booktitle = {The Stanford Encyclopedia of Philosophy},
    year      = {2021},
    editor    = {Zalta, Edward N.},
    publisher = {Metaphysics Research Lab, Stanford University},
    note      = {Substantive revision Tue Feb 2, 2021; available at: \url{https://plato.stanford.edu/archives/win2021/entries/qm-relational/}}
}

@article{DeAngelis2023,
    author    = {De Angelis, Valter and Rossi, Matteo and Grasso, Damiano and Zaffino, Riccardo L. and Fruci, Maria},
    title     = {{Is Science a Belief System? The Role of Scientists' Beliefs and Values in the Scientific Process}},
    journal   = {Foundations of Science},
    volume    = {28},
    pages     = {1565--1580},
    year      = {2023},
    doi       = {10.1007/s10699-022-09859-6}
}

@article{Facco2017,
    author  = {Facco, E. and Fracas, F. and Mento, G. and Tagliagambe, S. and Tressoldi, P.},
    title   = {{Dual-aspect monism and the deep structure of meaning}},
    journal = {Annals of the New York Academy of Sciences},
    volume  = {1389},
    number  = {1},
    pages   = {116--129},
    year    = {2017},
    doi     = {10.1111/nyas.13280}
}

@article{Atmanspacher2020,
    author  = {Atmanspacher, Harald},
    title   = {{The Pauli-Jung Conjecture and Its Relatives: A Formally Undecidable Proposition?}},
    journal = {Mind and Matter},
    volume  = {18},
    number  = {1},
    pages   = {9--20},
    year    = {2020}
}
\end{filecontents*}

% --- MAIN DOCUMENT ---
\documentclass[fleqn,journal,twoside]{IEEEtran}

% --- PACKAGES ---
\usepackage[utf8]{inputenc}
\usepackage[T1]{fontenc}
\usepackage{amsmath, amssymb, amsfonts, amsthm}
\usepackage{graphicx}
\usepackage{cite}
\usepackage[hyphens]{url}
\usepackage{hyperref}
\hypersetup{
    colorlinks=true,
    linkcolor=blue,
    filecolor=magenta,
    urlcolor=cyan,
    citecolor=blue,
    pdftitle={Project PRAETOR: Dominance in the Epistemological Battlespace through a Post-Materialist Quantum Sensorium},
    pdfauthor={Sunil Kukreja, Ph.D.},
    pdfsubject={UAP, Sensor Fusion, Quantum Metrology, Post-Materialism, National Security, Threat Assessment, Epistemology},
    pdfkeywords={PRAETOR, UAP, Sensor Fusion, Quantum Gravimetry, Post-Materialist Science, Paradigm Shift, National Security, Epistemological Battlespace},
}
\usepackage{microtype}
\usepackage{enumitem}
\setlist{nosep}
\usepackage{float}
\usepackage{fancyhdr}
\usepackage{booktabs}
\usepackage{siunitx}
\usepackage{array}

% --- Section Spacing ---
\usepackage[compact]{titlesec}
\titlespacing{\section}{0pt}{*2}{*1}
\titlespacing{\subsection}{0pt}{*1.5}{*0.5}
\titlespacing{\subsubsection}{0pt}{*1}{*0}

% --- Page Style using fancyhdr ---
\fancyhf{} % Clear all header and footer fields
\renewcommand{\headrulewidth}{0pt} % No header rule
\lfoot{\small AetherVision Paper ID\#: 2025-06-18/022~\copyright2025 AetherVision, LLC} % Set the left footer
\rfoot{\thepage} % Add page number to the right footer
\pagestyle{fancy} % Apply the fancy style to the document

% --- Math Environments ---
\newtheorem{definition}{Definition}
\newtheorem{theorem}{Theorem}
\newcolumntype{L}[1]{>{\raggedright\let\newline\\\arraybackslash\hspace{0pt}}m{#1}}
\newcolumntype{C}[1]{>{\centering\let\newline\\\arraybackslash\hspace{0pt}}m{#1}}
\newcolumntype{R}[1]{>{\raggedleft\let\newline\\\arraybackslash\hspace{0pt}}m{#1}}

% --- TITLE & AUTHOR ---
\title{Project PRAETOR: Dominance in the Epistemological Battlespace through a Post-Materialist Quantum Sensorium}

\author{
    \IEEEauthorblockN{Sunil Kukreja, Ph.D.}
    \IEEEauthorblockA{\textit{Founder \& CEO} \\
    \textit{AetherVision, LLC}\\
    Glastonbury, CT \\
    \href{mailto:sunil@aethervision-rd.com}{sunil@aethervision-rd.com} \\
    \href{https://aethervision-rd.com/}{https://aethervision-rd.com}}
}

\begin{document}

\maketitle
\IEEEdisplaynontitleabstractindextext
\thispagestyle{fancy}

\begin{abstract}
The challenge of Unidentified Anomalous Phenomena (UAP) represents a significant strategic threat and a profound crisis for the materialist paradigm that underpins Western science. Current national security sensors are inadequate for characterizing these phenomena, creating an unacceptable intelligence gap. This paper introduces Project PRAETOR (Praj\~n\=a Reconnaissance and Epistemological Threat Observation Resource), a persistent, multi-modal quantum sensorium designed to achieve information dominance in this new battlespace. PRAETOR integrates co-boresighted, high-speed multispectral imagers, spectrographs, passive RF analyzers, chiral polarimeters, and differential quantum gravity gradiometers to secure incontrovertible, multi-physics data on anomalous targets. We posit that the UAP challenge is not merely physical but epistemological—a battle over the nature of reality itself. By engineering an instrument to test the absolute limits of known physics, we are not only developing a critical tool for threat identification but are also directly probing the operational boundaries of materialism. PRAETOR is designed to transition the UAP problem from speculation to quantitative analysis, providing decisive evidence to either re-validate our current understanding of physics or compel a paradigm shift toward a post-materialist framework where consciousness and information are fundamental.
\end{abstract}

\begin{IEEEkeywords}
PRAETOR, UAP, Sensor Fusion, Quantum Gravimetry, Post-Materialist Science, Paradigm Shift, National Security, Epistemological Battlespace.
\end{IEEEkeywords}

\IEEEpeerreviewmaketitle

\section{Introduction}
\IEEEPARstart{T}{he} subject of Unidentified Anomalous Phenomena (UAP), once relegated to the cultural fringe, has been thrust into the forefront of national security discourse and scientific examination. This paradigm shift is driven by official government acknowledgments, notably from the U.S. Office of the Director of National Intelligence (ODNI) \cite{ODNI2021}, and is now systemized by the All-domain Anomaly Resolution Office (AARO) \cite{AARO2024} and a NASA independent study team \cite{NASA2023}. These accounts, substantiated by extensive testimony from decorated military aviators and sensor operators, describe phenomena that appear to violate the known principles of physics and aerospace engineering \cite{Kirkpatrick2021}. The reported performance characteristics—including extreme acceleration and sustained hypersonic velocities without expected thermal signatures—stand in stark contrast to the well-understood aerothermodynamic constraints of conventional hypersonic vehicles \cite{Blevins2019}.

This paper argues that the core obstacle to comprehending these phenomena is a profound and persistent lack of high-fidelity, multi-modal data. The current body of evidence, a fragmented collection of ambiguous and uncalibrated captures, fosters an epistemological stalemate between reflexive skepticism and unsubstantiated speculation. This is a classic example of a Kuhnian crisis, where anomalous observations that cannot be easily dismissed strain the dominant scientific paradigm \cite{Kuhn1962}. We posit that this impasse is not merely technological but is also philosophical, rooted in the limitations of a strictly materialistic worldview that may be ill-equipped to accommodate phenomena that radically subvert its foundational axioms \cite{Nagel2012}. By reflexively dismissing or failing to rigorously investigate such data, science risks behaving not as a method of inquiry but as a dogmatic belief system \cite{DeAngelis2023, Popper1959}.

To overcome this, we establish a territory for inquiry at the intersection of national security, advanced engineering, and the philosophy of science. We identify a critical niche: the need for an instrument purpose-built to collect incontrovertible, multi-physics data on UAP. This paper occupies that niche by introducing Project PRAETOR (\textbf{P}raj\~n\=a \textbf{R}econnaissance and \textbf{E}pistemological \textbf{T}hreat \textbf{O}bservation \textbf{R}esource). PRAETOR is designed to achieve dominance in what we term the `epistemological battlespace'—a domain of conflict where victory depends not just on physical assets, but on controlling the fundamental understanding of reality itself. It serves a dual role: firstly, as a crucial national security asset for threat identification, and secondly, as a profound epistemological tool for empirically testing the boundaries of scientific materialism and exploring the operational framework of a post-materialist science \cite{Beauregard2014, Wheeler1990}.

\section{Method: The PRAETOR Sensorium Architecture}
The proposed system is a deeply integrated suite of co-boresighted sensors governed by a real-time data fusion engine, designed to be reproducible and to allow for the critical judgment of its results.

\subsection{Multispectral Imaging and Spectrometry}
The system's `eyes' are a large-aperture ($>$\SI{0.5}{\meter}) Ritchey-Chr{\'e}tien telescope coupled with a real-time adaptive optics (AO) system employing pyramid wavefront sensors and high-speed deformable mirrors to achieve diffraction-limited imaging \cite{Guyon2018}. This choice is justified by the need to resolve small, distant objects against atmospheric turbulence.
\begin{itemize}
    \item \textbf{High-Speed Imaging Array:} Dichroic beam-splitters will channel light to three cryo-cooled, back-illuminated sCMOS detectors with documented sub-electron read noise at high frame rates \cite{Robbins2013}. The array will cover Visible/NIR (\SI{400}{\nano\meter}--\SI{1000}{\nano\meter}), SWIR (\SI{1000}{\nano\meter}--\SI{1700}{\nano\meter}), and MWIR (\SI{3}{\micro\meter}--\SI{5}{\micro\meter}) to provide a comprehensive thermal and visual signature.
    \item \textbf{Real-Time AI/ML Cueing:} Raw video feeds ($>$\SI{1}{gigapixel/sec}) are processed by a GPU cluster running a transformer-based model like Detection Transformer (DETR) \cite{Carion2020}, chosen for its end-to-end approach that eliminates the need for hand-crafted post-processing steps like non-maximum suppression.
    \item \textbf{High-Resolution Echelle Spectrograph:} Upon target lock, a flip-mirror directs light to a high-resolution ($R = \lambda/\Delta\lambda > 50,000$) Echelle spectrograph to search for non-terrestrial isotopic ratios or unexpected emission/absorption lines.
\end{itemize}

\subsection{Passive RF Intercept and Analysis}
A spherical phased array of software-defined radios (SDRs) ensures 360-degree surveillance from HF to Ka-band.
\begin{itemize}
    \item \textbf{High-Resolution DoA Estimation:} The system will implement subspace-based Direction of Arrival (DoA) algorithms. The MUSIC algorithm is selected for its high resolution with multiple signals, maximizing the pseudospectrum $P_{\text{MUSIC}}(\theta) = [a(\theta)^H E_n E_n^H a(\theta)]^{-1}$, where $a(\theta)$ is the steering vector and $E_n$ is the matrix of noise eigenvectors of the signal covariance matrix \cite{Schmidt1986}.
    \item \textbf{Advanced SIGINT:} The SDR backend will perform real-time cyclostationary analysis to identify and characterize Low-Probability-of-Intercept (LPI) waveforms that are invisible to standard energy detectors \cite{Gardner2006}.
\end{itemize}

\subsection{Chiral Polarimetry}
A subsystem will continuously measure the complete Stokes vector $S = [I, Q, U, V]^T$ of incident light.
\begin{itemize}
    \item \textbf{Search for Homochirality:} Detecting a distinct circular polarization signal ($V \neq 0$) is a powerful and agnostic remote biosignature, as biological processes on Earth exhibit a preference for one-handedness (homochirality) \cite{Patty2022}.
    \item \textbf{Probe for New Physics:} Detecting anomalous polarization rotations in the vacuum surrounding a UAP would be powerful evidence for physics beyond the Standard Model, such as the effects of axion-like particles \cite{Sikivie2021}.
\end{itemize}

\subsection{Differential Quantum Gravity Gradiometer}
This component employs field-deployable differential atom interferometry, representing the current state-of-the-art in deployable gravity sensing \cite{Bucar2022, Han2022}.
\begin{itemize}
    \item \textbf{Principle:} The core principle is the measurement of the phase shift $\Delta\Phi$ in a Mach-Zehnder-like atom interferometer, given by $\Delta\Phi = k_{\text{eff}} g T^2$, where $k_{\text{eff}}$ is the effective laser wavevector, $g$ is the local gravitational acceleration, and $T$ is the time between laser pulses \cite{Cronin2009}. The differential measurement between two spatially separated interferometers cancels common-mode noise, isolating local gravitational anomalies with high sensitivity.
    \item \textbf{Signature of Gravitational Manipulation:} This directly tests propulsion hypotheses. A technology manipulating the spacetime metric, such as an Alcubierre drive \cite{Alcubierre1994}, would require exotic matter that violates known energy conditions. While widely considered unphysical due to quantum inequality constraints \cite{Pfenning1997, Ford1998}, a positive detection would provide definitive evidence of new physics.
\end{itemize}

\subsection{Data Fusion and Bayesian Anomaly Detection}
To achieve rigor, the data fusion engine employs an Interacting Multiple Model (IMM) implementation of the Kalman filter \cite{Kalman1960}, with anomaly detection framed as a Bayesian model selection problem \cite{Jaynes2003, Cox1946}. The IMM algorithm runs multiple Kalman filters in parallel, each corresponding to a different motion model (e.g., constant velocity, coordinated turn, extreme non-ballistic acceleration). The overall state estimate is a probabilistic sum of the individual filter estimates, weighted by the likelihood of each model given the measurement history \cite{Bar-Shalom2001, Roy2017}. An anomalous behavior is defined as a dynamic not well-described by any of the pre-defined `known physics' models. We compare the anomalous model ($\mathcal{M}_{\text{anom}}$) to the best-fitting known model ($\mathcal{M}_{\text{known}}$) using the Bayes factor, $K = P(E | \mathcal{M}_{\text{anom}}) / P(E | \mathcal{M}_{\text{known}})$ \cite{Jeffreys1961}. An Anomaly Index $A_i = \log_{10}(K) > 2$ constitutes ``decisive'' evidence for the anomalous model \cite{Kass1995}.

\section{Results: Expected Signatures of Anomaly}
The sensorium is designed to produce unambiguous, multi-modal evidence. A positive detection would not be a single-channel anomaly but a cascade of correlated, physics-defying signatures across multiple sensor modalities. Table \ref{tab:signatures} summarizes key expected results that would provide decisive evidence for anomalous phenomena. The primary result of a successful detection campaign would be a dataset where the Bayes factor overwhelmingly favors an anomalous model, forcing a scientific reckoning.

\begin{table*}[ht]
\centering
\caption{Expected Signatures of Anomalous Phenomena from PRAETOR}
\label{tab:signatures}
\renewcommand{\arraystretch}{1.5}
\begin{tabular}{L{1.5in} L{2.5in} L{2.2in}}
\toprule
\textbf{Anomaly Class} & \textbf{Primary Observable Signatures} & \textbf{Implication / Model Falsified} \\
\midrule
\textbf{Kinematic / Gravimetric} &
\begin{itemize}
    \item Calculated acceleration $>500$ g with no sonic boom or thermal signature (from imagers).
    \item Localized, transient gravity gradient measured by the differential atom interferometer, correlated with the object's presence.
\end{itemize} &
Falsifies Newtonian/relativistic mechanics for macroscopic objects under conventional forces. Suggests manipulation of inertia or the local spacetime metric. \\
\addlinespace
\textbf{Electromagnetic / Spectral} &
\begin{itemize}
    \item Strong EM emissions with no terrestrial modulation scheme (from SDR).
    \item MWIR signature inconsistent with blackbody radiation for observed velocity (violates expected aerothermodynamics).
    \item High-resolution spectrum showing non-terrestrial isotopic ratios (e.g., in magnesium or other light metals).
\end{itemize} &
Falsifies terrestrial origin. An acausal thermal signature would falsify the standard model of atmospheric friction and heat dissipation. \\
\addlinespace
\textbf{Exotic / Quantum} &
\begin{itemize}
    \item Detection of circular polarization ($V \neq 0$) from an apparently sterile, metallic object (from polarimeter).
    \item Measurable rotation of polarized light in the vacuum surrounding the object.
    \item Apparent violation of causality (e.g., RF signal received before optical detection of maneuver).
\end{itemize} &
Agnostic biosignature (chiral signal). Falsifies the vacuum of standard QED (polarization rotation). Suggests advanced quantum effects, possibly involving retrocausality or non-local interactions. \\
\bottomrule
\end{tabular}
\end{table*}

\section{Discussion}
Evidence does not explain itself. The purpose of this discussion is to explain the potential results and show how they would help answer the foundational questions posed in the Introduction, moving from the specific results to their general implications.

\subsection{The Incompleteness of Scientific Materialism}
The acquisition of data as described in Table \ref{tab:signatures} would represent a first-order crisis for scientific materialism. This philosophical stance, which asserts that matter is the fundamental substance of nature and that all phenomena, including consciousness, are results of material interactions, has been the engine of unprecedented progress. Yet, its most thoughtful critics have long pointed to its profound explanatory gaps. These include the ``hard problem'' of consciousness—why and how material processes should give rise to subjective experience \cite{Chalmers1995}—and the general failure of the materialist neo-Darwinian framework to account for the existence of mind and reason \cite{Nagel2012}. UAP represent a physical, rather than purely philosophical, challenge to this worldview. If objects can demonstrably violate the physical laws derived from a materialist framework, then the framework itself must be considered incomplete. A dogmatic refusal to engage with such data, as Popper warned, transforms science from a method of discovery into an unfalsifiable ideology \cite{Popper1959, Hossenfelder2018}.

\subsection{Quantum Mechanics as the Bridge to Post-Materialism}
Quantum mechanics, the most successful and well-tested theory in the history of science, provides the formal foundation for transcending classical materialism \cite{Gisin2021}. Its core tenets are inherently non-materialist in the classical sense:
\begin{itemize}
    \item \textbf{The Violation of Local Realism:} Initiated by the EPR paradox \cite{EPR1935}, codified by Bell's theorem \cite{Bell1964}, and conclusively validated in loophole-free experiments \cite{Hensen2015}, quantum mechanics has proven that reality is fundamentally non-local. An interaction with one particle can instantaneously influence its entangled partner, regardless of the distance separating them. This wholeness is a direct contradiction of the classical materialist view of separate, interacting objects.
    \item \textbf{The Primacy of Information and the Observer:} Recent theoretical work suggests that quantum theory cannot even be used to consistently describe itself when applied to complex systems including observers, leading to paradoxes that challenge the notion of an objective, observer-independent reality \cite{Frauchiger2018}. This has fueled the rise of observer-centric frameworks. \textbf{Relational Quantum Mechanics (RQM)}, for example, posits that a system's properties are not intrinsic but are only meaningful in relation to another system (the observer). There is no absolute, ``God's-eye'' view of reality; there are only interactions \cite{Rovelli1996, Rovelli2021}. Similarly, QBism and the participatory observer models of Wheeler and Stapp argue that a quantum measurement is an action on the world by an agent, and the outcome is the agent's experience \cite{Wheeler1990, Stapp2011}.
    \item \textbf{Quantum Biology:} The once-firm boundary between the quantum and classical worlds is dissolving. Evidence continues to mount for non-trivial quantum effects in warm, wet biological systems, such as in photosynthesis \cite{Engel2007}, avian navigation, and potentially even in brain function via mechanisms in microtubules \cite{Hameroff2014, Andre2024}. This demonstrates that the strange principles of quantum mechanics are not confined to the subatomic realm but are integral to the workings of life and consciousness, directly challenging the materialist assumption that quantum effects `wash out' at the macroscopic scale \cite{Tegmark2000}.
\end{itemize}

\subsection{Consilience with Ancient Philosophies}
Interestingly, some concepts at the heart of the post-materialist debate find deep echoes in ancient philosophical traditions, particularly the Hindu Vedas and Upanishads. While mainstream academic consensus dates the composition of the earliest Veda, the Rigveda, to approximately 1500--1200 BCE, it is important to note these traditions represent millennia of philosophical inquiry. Claims of far greater antiquity, such as 12,000 years, lack corroborating scientific or archaeological evidence but highlight the enduring power of these ideas. To be clear, these are spiritual and metaphysical texts, not scientific treatises. However, their conceptual frameworks bear a striking resemblance to insights emerging from modern physics.
\begin{itemize}
    \item \textbf{Brahman and M\=ay\=a:} The Advaita Vedanta school posits \textit{Brahman} as the ultimate, singular, and unchanging reality, an undifferentiated field of pure consciousness \cite{Dasgupta1922}. The phenomenal world we perceive is considered \textit{M\=ay\=a}—not strictly an illusion, but a projection, appearance, or dependent reality, shaped by the act of observation. This resonates deeply with relational interpretations of quantum mechanics, where the physical world, at its most fundamental level, is not a collection of solid objects with pre-existing properties, but a web of relations that comes into being through interaction \cite{Rovelli2021}.
    \item \textbf{Interconnectedness and Consciousness:} The Vedic concept of a unified, interconnected cosmos, where all beings are manifestations of the singular \textit{Brahman}, is a conceptual parallel to quantum entanglement \cite{Schlosshauer2019}. Furthermore, the idea that consciousness is not a passive observer but an active participant in the manifestation of reality is central to these traditions \cite{Radhakrishnan1953}. This aligns with observer-centric quantum models and with modern dual-aspect monist theories that treat mind and matter as two sides of the same fundamental reality, which itself may be neutral and psychophysical in nature \cite{Facco2017, Atmanspacher2020}.
\end{itemize}

\section{Conclusions}
The study of UAP has been trapped in a feedback loop of inadequate data and prohibitive dogma. To shatter this cycle, we have detailed a concrete technological solution: Project PRAETOR, a multi-domain, multi-physics sensorium capable of generating a rich, calibrated, and undeniable dataset. This system is grounded in the mathematical rigor of optimal state estimation and Bayesian inference, and is engineered to substitute quantification for ambiguity. It provides an indispensable asset for national security and a powerful tool for fundamental physics.

However, the ultimate thesis of this paper is that the UAP problem is not merely a technological challenge but a profound philosophical one, fought in the epistemological battlespace. A dogmatic adherence to a narrow, materialistic worldview has rendered mainstream science ill-equipped to investigate what may be the most significant discoveries of our era. The act of building an instrument to systematically search for evidence that contradicts our dominant paradigm is, therefore, the most rigorously scientific course of action. It directly addresses the research questions of whether our current physical and philosophical models are complete.

The implications of success are difficult to overstate. The PRAETOR sensorium is an epistemological instrument—a probe designed to pierce the veil of \textit{M\=ay\=a}, the world of appearances. It is a declaration of our ignorance and an assertion of our profound commitment to learn. By seeking to measure the ``impossible,'' we are not engaging in pseudoscience; we are undertaking the foundational act of all great scientific revolutions: to question our deepest assumptions and dare to see what lies beyond the horizon of the known. The data from this system may not only resolve the mystery of UAP; it may provide the first empirical evidence to compel the development of a new, post-materialist conception of reality itself, one in which information is physical, the observer is central, and consciousness is fundamental.

\section*{Acknowledgments}
The author extends his gratitude to the pioneers of quantum metrology, sensor engineering, and those thinkers, both ancient and modern, who have had the courage to question the boundaries of our perceived reality.

\bibliographystyle{IEEEtran}
\bibliography{Quantum-Reality}

\end{document}